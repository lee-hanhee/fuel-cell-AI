The database review serves as an overview of the review conducted before my project, which is organized into 6 columns: title, author(s), year, keywords, main findings, and the relevance to the project. This has 4 distinct types of data that will be color-coded: literature (red), code (blue), data (orange), miscellaneous (green), in which literature will be the only type of data to summarize the main findings using the CRAAP test. The data will be hyperlinked (if possible) or have the corresponding file name \& folder, which can be referenced using the readme file. Any of the notes that I create may not be put into the database.  
\section{Research trends}
    
\section{Relevant literature}

    \newpage \begin{table}[H]
    \centering
    \begin{tabularx}{\textwidth}{HIJ} % Custom column widths
    \toprule
    \textbf{Title} & \textbf{Author} & \textbf{Summary} \\ 
    \midrule
    A Review of Machine Learning Techniques for Predictive Maintenance & Wang et al. & The paper provides a comprehensive review of machine learning techniques for predictive maintenance, including supervised and unsupervised learning methods, deep learning, and reinforcement learning. It discusses the challenges and opportunities in applying these techniques to predictive maintenance and highlights the importance of data quality, feature selection, and model interpretability. The paper also presents case studies and applications of machine learning in predictive maintenance across various industries. \\

    \bottomrule
    \end{tabularx}
    \caption{Summary of studies on predicting PEMFC performance metrics using machine learning}
    \end{table}